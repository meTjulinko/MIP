% Metódy inžinierskej práce

\documentclass[10pt,oneside,slovak,a4paper]{article}
\usepackage{pdfpages}
\usepackage{graphicx}
\usepackage[slovak]{babel}
%\usepackage[T1]{fontenc}
\usepackage[IL2]{fontenc} % lepšia sadzba písmena Ľ než v T1
\usepackage[utf8]{inputenc}
\usepackage{graphicx}
\usepackage{url} % príkaz \url na formátovanie URL
\usepackage{hyperref} % odkazy v texte budú aktívne (pri niektorých triedach dokumentov spôsobuje posun textu)
\usepackage{multirow}
\usepackage{cite}
\usepackage{makecell}

%\usepackage{times}

\pagestyle{plain}

\title{Hry a ich efekt na mentálne zdravie\thanks{Semestrálny projekt v predmete Metódy inžinierskej práce, ak. rok 2022/23, vedenie: Vlado Mlynarovič}} % meno a priezvisko vyučujúceho na cvičeniach

\author{Matúš Hrkeľ\\[2pt]
	{\small Slovenská technická univerzita v Bratislave}\\
	{\small Fakulta informatiky a informačných technológií}\\
	{\small \texttt{xhrkelm@stuba.sk}}
	}

\date{\small 6. november 2022} % upravte



\begin{document}

\maketitle

\begin{abstract}
V tejto dobe sú počítačové hry neodmysliteľnou časťou našich životov. Každý si ich zapne minimálne raz do týždňa. V tomto článku by som sa chcel zamerať na ich efekt na naše mentálne zdravie. Pozrieť sa na negatívne ale zároveň aj na ich pozitívne efekty. Či nám dokážu pomôcť so stavmi ako depresia, úzkosti, ADHD, osamelosť a mnohé iné. Taktiež ako nás ovplyvňujú, keď sa staneme na nich závislí a trávime pri nich príliš veľa času. Či je možné nadobudnúť nejaké sociálne alebo iné zručnosti. Taktiež by som sa pozrel ako vyzerali hry v minulosti, a ako sa menil ich efekt na mentálne zdravie. Toto všetko by som zhodnotil a povedal môj pohľad na vec.
\end{abstract}
\newpage

\section{Úvod}

Pojem mentálne zdravie sa v tejto dobe spomína čoraz viac a viac. Počas pandémie Covid-19 a lockdownu sme boli zavretí v našich domovoch celé dni. Učili sme sa online a vôbec sme sa nestretávali s ostatnými ľuďmi a nezažívali sme sociálne interakcie. 

Tento stav mal na naše mentálne zdravie určite nejaký ten efekt. Veľa ľudí prepadlo depresiám alebo úzkostiam. Osamelosť pociťoval skoro každý, kto bol zavretý celé dni doma. Práve cez toto obdobie si veľa ľudí skracovalo čas hraním videohier, pri ktorých mohli naviazať kontakt s ostatnými ľuďmi. Cez túto dobu hlásili herné služby rekordné počty užívateľov a počet odohraných hodín na ich hrách. \cite{steam}

Herný svet sa stal každodennou realitou nespočetného kvanta ľudí. Aký to však malo efekt? Jedna strana hovorí, že ľudia prepadli závislostiam, stali sa násilnými a hry im škodili a druhá, že práve hry im zlepšovali náladu a ich psychické zdravie. 



\section{Mentálne zdravie} \label{Mentalne Zdravie}

Všade sa spomína pojem mentálne zdravie, ale len pár ľudí ozaj vie čo to znamená. Je to naša emociálna, psychická a sociálna pohoda. Ovplyvňuje to ako sa cítime, na čo myslíme a ako sa správame. Taktiež hrá veľmi dôležitú rolu pri tom ako zvládame stres a ako sa rozhodujeme. 

Je to veľmi dôležitý aspekt nášho života a preto by sme sa mali oň starať ako o fyzické zdravie. Keďže je dôležité pri našom vývoji, od detstva by sme si mali dať pozor ako sa oň staráme a čo do seba púšťame. \cite{CDC}




\subsection{Duševné choroby} \label{Mentalne Zdravie- Choroby}

Najčastejšie duševné choroby, ktoré sa vyskytujú vo svete sú depresia a úzkosť.  Ovplyvňujú viac ako 14\% globálnej populácie. Predpovedá sa, že do roku 2030 sa stanú jednou z troch najčastejších príčin smrti na svete. Lockdown a pandémia rozšírili výskyt týchto chorôb vo svete signifikantným spôsobom. Takmer žiadny kontakt s vonkajším svetom a nulová socializácia boli najväčšie príčiny vzniku týchto chorôb. 

Depresia a úzkosť sa zvyčajne vyskytujú spoločne, ale taktiež sú aj často spojované s chorobami ako osamelosť alebo iné. V spoločnosti panuje pocit, že o príčinách a symptómoch týchto chorôb a stavov sa veľmi nemá rozprávať. Práve kvôli tomuto veľa ľudí nevyhľadá pomoc, lebo si myslia, že si budú ostatný o nich niečo namýšľať. Taktiež ak niekto potrebuje pomoc, tak ju nemusí dostať lebo na to nie sú zdroje. Preto sa viacej ľudí schyľuje ku videohrám, ktoré im pomáhajú v ich situácií a pri ktorých sa vedia odreagovať.\cite{Kowal:Mitigation}


\section{Video hry ako nástroj na pomoc} \label{Videohry}

Videohry slúžia primárne ako nástroj na relax a zábavu. Mnohí si pri nich krátia svoj čas. Sú široko dostupné a stali sa veľkou časťou modernej spoločnosti. Každá domácnosť má doma minimálne jednu videohru. Ich všadeprítomnosť môže byť opodstatnená hlavne ich zábavnou stránkou a ich širokou dostupnosťou. Dnes je možne hrať hry na našich počítačoch, smartfónoch, konzolách, televízoroch dokonca aj na hodinkách. Hry sú cenovo v rozmedzí od tých čo sú zadarmo až po tie čo stoja niekoľko desiatok eur.\cite{share}

	Od ich počiatku sa predpovedalo, že majú na naše mentálne zdravie nejaké pozitívne efekty. Postupne sa objavovalo väčšie množstvo dôkazov, ktoré tieto tvrdenia potvrdili. Prišlo sa na to, že hry sú vysoko prospešné v oblasti socializácie, spoznávania sveta, regulácie emócií a mentálneho zdravia. Vzhľadom na toto a čoraz väčším dopytom po starostlivosti o mentálne zdravie, sú hry čím ďalej viacej používané na liečenie mentálneho zdravia. \cite{Benefits}

\subsection{Dôkazy, že hry podporujú mentálne zdravie} \label{Videohry}

Potenciálne účinky boli skúmane na klinickej populácií. Skúmali sa napríklad účinky na ľuďoch, ktorí trpeli buď depresiou, úzkosťami, schizofréniou a inými. Použili sa videohry žánru RPG(hry na hranie rolí/hrdinov),simulácie, akčné hry, hry pre viac hráčov a iné.

V nasledujúcom zozname uvádzam daný výskum, ktorý zdôrazňuje potenciál videohier na zmiernenie príznakov najrozšírenejších porúch duševného zdravia: depresia,osamelosť a úzkosť.\\

\begin{flushleft}
\hspace*{-2,1cm}
\begin{tabular}{ |p{5cm}||p{5,2cm}|p{5cm}| }
 \hline
 \multicolumn{3}{|c|}{\textbf{ \makecell{Zoznam študií zaoberajúcich sa videohramy a ich \\potenciálom na zmiernenie depresie a úzkosti:}}} \\
 \hline
 \textbf{Sledovaný aspekt mentálneho zdravia}&\textbf{Hra alebo žáner}&\textbf{Výsledok}\\
 \hline
 Prosociálne správanie a znížená osamelosť   & AH*,RPG*, hry pre viac hráčov    &Priniesli benefity v oblasti socializácie\\
\hline
 Vedomosti&   AH,strategické hry,športové hry  &Zlepšenie vedomostí, dokonca niektoré zmiernili dyslexiu   \\
\hline
 Cieľavedomosť &Portal2(VH*),TeamFortress2(VH), RPG & Zlepšili ciaľavedomosť a motiváciu dosiahnuť svoje ciele\\
\hline
 Nálada    &Portal2(VH),MarioKart(VH),RPG & Zlepšenie nálady v čase aj veľkosti \\
\hline
 Regulácia emócií& Flappybird(VH),Slenderman(VH), celkové hranie hier  & Uľahčovali zvládanie silných emócií a reguláciu silných emotívnych zážitkov \\
\hline
 Depresívna nálada& CandyCrush(VH),Angry Birds(VH),Limbo(VH)  & Znížili negatívne emócie propagáciov radosti a zlepšili motiváciu   \\
\hline
 Všeobecná úzkosť& Strategické hry, RPG, MindLight(VH)  & Znížili značným množstvom úzkosť a opakované hranie jej pomáhalo predísť \\
\hline
 Prevencia úzkosti& Rayman(VH) &Podobným spôsobom ako pri všeobecnej úzkosti znížili úroveň úzkosti\\
\hline
 Stavy úzkosti& Plantsvs.Zombies(VH), Peggle(VH),príležitostné hry& Znížili pocit úzkosti odreagovaním \\
 \hline
\end{tabular}
\end{flushleft}
*AH: Akčné hry, *RPG: Hry na hranie rolí/hrdinov, *VH: Videohra				.\cite{Kowal:Mitigation}\\

\section{Negatívne efekty videohier - závislosť} \label{ina:neg}

Video hry majú na nás veľa prospešných efektov. Avšak keď sa im venujeme príliš veľa tak sa môžeme stať na nich závislý. Ešte viacej sa to týka online hier pri ktorých ju nemôžeme len tak zastaviť. Pre veľa ľudí, keď sa spomenie pojem závislosť tak si predstavia branie drog a gembling v herniach. Do popredia sa však dostavajú čoraz viac veci, pri ktorých nemusíme nič brať. 

Aktivita, pri ktorej trávime veľmi veľa času a závislosť sú dve rozličné pojmy. Napríklad bicyklovanie, kde vieme stráviť aj celý deň nám pridáva do života. Závislosť je niečo čo nám z neho berie. Videohry sa často prirovnávajú ku herným automatom. Na tých hráme o peniaze, ktoré môžeme kedykoľvek stratiť. V hrách ide skôr o nejaké body. Podstata je však rovnaká. Preto sa dostáva závislosť na hrách do popredia. \cite{addiction}
\subsection{Príznaky}

Do dnes bolo urobených len veľmi málo výskumov, ktoré sa týmto problémom zaoberali. Všetky boli zamerané na adolescentov alebo mladých dospelých. Medzi najčastejšie príznaky závislosti patrili: kradnutie peňazí na hranie hier alebo na kúpu nových hier, záškoláctvo s úmyslom hrať hry, nerobenie si domácich úloh a dostávanie zlých známok v škole, vynechávanie sociálnych aktivít v prospech hrania, nespokojnosť a podráždenosť ak nebola možnosť na hru a hranie dlhšie ako jedinec chcel. 

Tieto príznaky nám napovedajú, že nejaká časť mladých ľudí je závislých na videohrách. Prečo sa však ľudia stávajú závislí a čo tomu najviac napomáha ešte potrebuje čas na zistenie. 
\subsection{Následky}

Každá zo závislosti ma nejaké následky na človeka. To isté platí aj pre videohry. Môžeme ich deliť na psychologické a fyzické. 

Medzi psychologické môžeme zaradiť: pocit eufórie a pohody pri trávení času na počítači, nemožnosť zastaviť aktivitu, nutkanie na trávenie viac a viac času pri počítači, zanedbávanie rodiny a priateľov, klamanie ostatným o aktivitách a problémy so školou alebo prácou. Väčšina z týchto psychických následkov nám môže pripomínať závislosť na drogách. Jedinec má problém fungovať v reálnom svete. Jeho dennodenným životom sú hry a počítač. 

Existujú aj fyzické následky závislosti. Medzi nie patria: karpálny tunel, suchosť očí, migrény, bolesti chrbtice, problémy s jedením, zanedbávanie hygieny a zmeny v spánku. Z tohto vyplýva, že človek závislí na videohrách si môže aj dosť ublížiť. Doživotné problémy s chrbticou alebo karpálny tunel sú veľmi bolestivé choroby, ktoré obmedzovať v staršom veku.

\subsection{Online hry}

Tento typ hier má svoju vlastnú problematiku. Podľa všetkého sú oveľa návykové ako offline hry(hry pre jedného hráča). Jeden z deviatich hráčov online videohier spĺňa aspoň tri podmienky závislosti, teda je závislí na hrách. Podmienky boli definované WHO(Svetová zdravotnícka organizácia) a zahŕňajú nutkanie, toleranciu, abstinenčné príznaky, stratu kontroly, zanedbávanie ostatných činností a ďalšie negatívne prvky. Taktiež hrali videohry podstatnejšie dlhšie ako hráči offline hier. Takisto výskyt abstinenčných príznakov a nutkanie sa vyskytovali viac. Preto sú online hry, ktoré ponúkajú akciu 24 hodín denne značne nebezpečnejšie a návykové. 
\subsection{Pomoc so závislosťou}

Liečba závislosti od videohier je podobná ako u iných závislostiach. Medzi najzákladnejšie praktiky patrí poradenstvo s psychológom a zmena správania a spôsobu života. Poradenstvo spojené s rodinou môže mať väčší efekt. Niektoré liečebné zariadenie dokonca predpisujú aj lieky v ich programoch. 

Avšak na rozdiel od drog a alkoholu, videohry sa hrajú na počítačoch, ktoré potrebujeme používať každý deň. Abstinencia teda nie je možná pri všetkých prípadoch.  Toto je veľmi podobné ako závislosť na jedle – obezita. Preto je veľmi dôležitá kontrola a sebaovládanie na zmenu. 

Neexistuje žiadny univerzálny spôsob, ktorý ju dokáže vyliečiť. Taktiež je možnosť spojiť sa online s inými ľuďmi, ktorí si prechádzajú závislosťou a rozprávať sa o tom čo im pomohlo a podporovať sa.  
\section{Zručnosti nadobudnuté pomocou videohier}
\dots
\section{História} \label{History}
\dots


\section{Záver} \label{zaver} % prípadne iný variant názvu
\dots


%\acknowledgement{Ak niekomu chcete poďakovať\ldots}


% týmto sa generuje zoznam literatúry z obsahu súboru literatura.bib podľa toho, na čo sa v článku odkazujete
\bibliography{literatura}
\bibliographystyle{plain} % prípadne alpha, abbrv alebo hociktorý iný
\end{document}
